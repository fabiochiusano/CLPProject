There are a lot of possible ways to extend our compiler for graphic and user interaction.
For instance, we can:
\begin{itemize}
\item{}
add new graphic native methods by simply adding them to ToolGUI.java and GUI.tool, such as some for choosing colors and filling shapes. That would imply then to recompile ToolGUI.java;
\item{}
extend the Tool graphic library with more high-level classes and functions with more complex geometric shapes and actions, such as an action that makes a shape move in circular motion, as shown in example 3. In order to implement complex shapes, using an array of \textit{Point} would be the easiest way but two arrays of integers (i.e. one for abscissas and one for ordinates) can be used too;
\item{}
add a float type, that would make easier to implement complex computations without relying on the \textit{Frac} class;
\item{}
add primitives to display bitmaps.
\item{}
extend the language to allow functions to return \textit{Unit}. That would make the code much more cleaner.
\end{itemize}